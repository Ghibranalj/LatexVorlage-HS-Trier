\chapter{Problemstellung}

Hier wird i.d.R. zunächst das generell vorliegende Problem diskutiert: Was ist zu lösen - was gibt es bisher an Lösungsansätzen (prinzipiell) und warum ist es wichtig, dass man dieses Problem löst. Letzteres ergibt sich oftmals aus der vorliegenden Anwendungssituation: Man braucht die Lösung, um eine bestimmte Aufgabe zu erledigen, ein System aufzubauen etc. Der Bezug auf vorhandene oder auch bisher fehlende Lösungen begründet auch die Intension und Bedeutung dieser Arbeit. Dies können allgemeine Gesichtspunkte sein - man liefert einen Beitrag für ein generell erkanntes oder zu erkennendes Problem - oder man hat eben eine spezielle Systemumgebung oder Produkt (z.B. in einer Firma u.s.w.), woraus sich dieses noch zu lösende Problem ergibt.

Die genaue Problematik und Randbedingungen werden dann in Kapitel \hyperref[Aufgabenstellung]{Kapitel~\ref{Aufgabenstellung}}
dargestellt.
